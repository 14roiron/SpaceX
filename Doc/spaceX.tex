\documentclass{scrartcl}

\usepackage[utf8]{inputenc} % Unicode support (Umlauts etc.)
\usepackage{hyperref} % Add a link to your document
\usepackage{graphicx} % Add pictures to your document
\usepackage{listings} % Source code formatting and highlighting
\usepackage[top=75px, bottom=75px, left=85px, right=85px]{geometry} % Change page borders
\usepackage{mathtools}
\usepackage{color}
%footer
\usepackage{fancyhdr}
\usepackage{enumitem}

%Packages
\usepackage[parfill]{parskip}
\usepackage{tikz}
%Configure Packages
\usetikzlibrary{automata, arrows, positioning}


\renewcommand{\headrulewidth}{0pt}
\renewcommand{\footrulewidth}{0.4pt}

\begin{document}

\title{Explanations of the program}
\date{\today{}}

\author{Roiron Yohann yohann.roiron@mines-paristech.fr}

\maketitle
%\tableofcontents \pagebreak

\section{Physics}
We are going to modelise the system as a dual body, one is fix: The earth, and the second one is moving, the rocket we are sending into space.
The aim of this programm is to simulate the launch ( A very simplified one, no body separation, to booster, and no rotations issue) and recuperation of the module. We will also Simulate the moon.

$$m\frac{d\vec V}{dt}=\vec u+ \vec F_{Earth \rightarrow Sat} + \vec F _{Moon \rightarrow Sat}$$
We can Discretise the system:

$$\vec V(t+1) - \vec V(t)=\frac{1}{m}\vec u(t)+ \frac{1}{m}\vec F_{Earth \rightarrow Sat}(t) + \frac{1}{m}\vec F _{Moon \rightarrow Sat}(t)$$

\end{document}